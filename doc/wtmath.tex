\documentclass[a4paper]{article}

\usepackage{enumitem}
\usepackage{shortvrb}
\usepackage{wtmath}

\setlist[description]{font=\normalfont}
\MakeShortVerb{\|}
\newcommand{\PkgName}{\textsf{WTMath}}
\newcommand{\Meta}[1]{$\langle$\mbox{}{\normalfont\textit{#1}}\mbox{}$\rangle$}
\newenvironment{syntax}{\begin{quote}\small}{\end{quote}}
\newcommand{\cmd}[1]{\texttt{\def\{{\char`\{}\def\}{\char`\}}\texttt{\symbol{'134}}#1}}

\makeatletter
\let\halmos\wtmt@halmos
\makeatother

\title{{\PkgName} Package (dev)}
\author{Watson}

\begin{document}

\maketitle

\begin{abstract}
WT Series collects macros which author frequently use to create {\LaTeX} documents.
{\PkgName} package is a part of this WT Series which contains a lot of macros for mathematics.
{\LaTeXe} on any kind of {\TeX} engine is supported. Moreover \textsf{amsmath}, \textsf{xkeyval}
package is required.
\end{abstract}

\section{System Requirements}

System requirements of {\PkgName} are shown bellow:
%
\begin{itemize}
\item {\TeX} engine: any engine
\item {\TeX} format: \LaTeXe
\item Document class: any class
\item Required package: \textsf{amsmath}, \textsf{xkeyval}
\end{itemize}

\section{Loading the {\PkgName} Package}

To use {\PkgName} package, load \texttt{wtmath.sty} file with |\usepackage| command in preamble.
No package option is available.
%
\begin{syntax}
|\usepackage{wtmath}|
\end{syntax}

\section{Basic commands}

{\PkgName} package defines some basic commands at the time you load it. It also redefine
part of \textsf{amsmath} commands and extends their functions.

\subsection{Commands for mathematics}

The package defines following commands for mathematics as default. These commands can be
used only in math mode unless otherwise noted.
%
\begin{description}
\item[\cmd{func\{\Meta{function name}\}}]
This command print function name.
%
\item[\cmd{eqsep}]
Put space between equations.
%
\item[\cmd{then}]
Output symbol ``$\then$''.
%
\item[\cmd{st}]
Output string ``such that''. Spaces are put arround it.
%
\item[\cmd{tand}]
Output string ``and''. Spaces are put arround it.
%
\item[\cmd{tor}]
Output string ``or''. Spaces are put arround it.
%
\item[\cmd{defeq}]
Output symbol ``$\defeq$'' which means define equation.
%
\item[\cmd{defiff}]
Output symbol ``$\!\!\defiff\!\!$'' which means define equivalence.
%
\item[\cmd{qed}]
Output halmos letter which means define equation. This command can be used in both
inner and outer math mode. Note that there are not equational number if use this command
in math mode.
\end{description}

\subsection{Overwrite \textsf{amsmath} commands}

Following commands, which are defined in \textsf{amsmath} package, redefined if you
load {\PkgName} package.
%
\begin{description}
\item[\cmd{bar\{\Meta{commands}\}}]
Put overline on \Meta{commands}. For example,\\
|\bar{A\times B}| outputs $\bar{A\times B}$.
\end{description}

\subsection{Commands for macros}

%以下のマクロは文書中に直接書き込むためというよりも,数式用のマクロを本パッケージのユーザが
%新たに定義する際に利用することを意図して定義されているものである.
%%
%\begin{description}
%\item[\cmd{relmiddle\{\Meta{symbol}\}}]
%指定した\Meta{symbol}を括弧類の間で使用される関係演算子として出力する.すなわち\TeX
%の |\mathrel| と\LaTeX の |\middle| を合わせたような働きをする命令である.数式モードでのみ
%使用可能.
%%
%\item[\cmd{mathbold\{\Meta{commands}\}}]
%数式モード内で,引数に与えた\Meta{commands}の内容を斜体の太字(ボールド体)で出力する.
%%
%\item[\cmd{exchangecmd\{\Meta{command 1}\}\{\Meta{command 2}\}}]
%与えられた2つの命令の定義を入れ替える.
%\end{description}

\section{Loading libraries}

%\PkgName は,パッケージを読み込むだけではすべての命令が使用可能な状態にはならない.
%マクロはその使用分野ごとにライブラリという形でまとめられ,本パッケージの使用者は必要に
%応じてライブラリを読み込むことで,それらの命令が利用可能になる.
%
%ライブラリの読み込みは |\usemathlibrary| 命令によって行う.
%%
%\begin{syntax}
%|\usemathlibrary{|\Meta{libraries}|}|
%\end{syntax}
%%
%ここで\Meta{libraries}にライブラリ名をカンマ区切りで記述することで,一度に複数のライブラリを
%読み込むことも可能である.この命令はプリアンブル以外でも使用可能で,|{| や |}| によるスコープの制御を受ける.

\section{Details of each libraries}

%\subsection{analysis}
%
%\textsf{analysis}ライブラリは極限や微分・積分に関わる命令を集めたものである.
%
%\subsection{lambda}
%
%\textsf{lambda}ライブラリはラムダ計算に関わる命令を集めたものである.
%
%\subsection{set}
%
%\textsf{set}ライブラリは集合に関わる命令を集めたものである.

\end{document}
